The Journal was founded in the 19th century, was active throughout the 20th century and in
1996 launched an online version. It has had multiple data sources (paper receipts, paper journals, text files, proprietary binary file databases and lately SQL databases) for the print newspaper over the years which have been integrated into one relational database. The online edition was designed separately and has its own databases. Both databases store customer data differently which makes it hard to aggregate statistics from both editions. Also, the database for the printed edition was built additively and contains data inconsistencies. The data warehouse would allow to combine personalized knowledge of the online edition readers with a lot of additional historical data from the printed edition.

WSJ is the largest newspaper in the United States by circulation and has more than 400,000 online paid subscriptions. As such, the amount of data accumulated over the years is huge and was not designed to be stored for efficient analysis. As a result, it’s very inefficient to extract useful information as the business queries would take a long time when run on production databases.

