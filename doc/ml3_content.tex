The main purpose of this milestone is to try out various data mining techniques.

We decided to explore the following data mining techniques in this milestone: classification, prediction and clustering. For each technique we evaluated various algorithms mentioned in the lectures and compared them to some others that were available to use in the data mining software package that we used. 

We also extracted some new information from the models built during the mining process to show some things that we found interesting which is also the goal of the MSR 2012 challange.

\section{Data} % (fold)
\label{sub:Data}
We chose to mine the Android bug database because it was the data was already provided by the organizers of the MSR 2012 challange. More specifically, we chose to focus on the bug reports in this database as we believe it is the most important part of the entire database and it should give the most interesting results. The database contains 20169 reported bugs (up to December 3, 2011). 

\subsection*{Description of the data} % (fold)
\label{sub:Description of the data}

Here is the available data about the bugs.

\begin{itemize}
\item Bug ID - unique identifier assigned to each bug
\item Title - short description of the bug
\item Description - detailed description of the bug without a predefined structure
\item Type - type of the bug: Defect (bug report), Enhancement (feature request)
\item Status - current status of the bug (Reviewed, New, Duplicate, Declined, NeedsInfo, FutureRelease, Released, Spam, Unreproducible, Question, WorkingAsIntended, Assigned, Unassigned, UserError)
\item Priority - urgency of the bug to be fixed (Small, Medium, High, Critical, Blocker)
\item Component - category of the bug based on which part of the system the bug affects
\item Stars - number of people that have starred this bug (could be an indicator of popularity)
\item Owner - assigned developer to fix  the bug
\item Reported By - person who reported the bug
\item Opened Date - when the bug was reported
\item Closed On - when the bug was marked as closed
\end{itemize}

It is also possible to aggregate the comments for a bug to determine additional data about it.

\begin{itemize}
\item Comment count - the number of comments for this bug
\item Number of commenters - the number of distinct people participating in the discussion about the bug
\item How long it took the solve the bug - by subtracting the close date from the open date
\end{itemize}

Title, description, type are entered by the user. System also automatically generated the bug id, marks it as a new bug, assigns a medium priority to the bug, makes the user as the reporter of the bug and sets the opened date to the date and time of the submition.

Component and owner are later manually assigned by one of the administrators. Status, priority can later be changed, however, these changes are not reflected in the available data, only the latest state of the bug is available.
% subsection Description of the data (end)

\subsection*{Processing of the data} % (fold)

The bug database which was provided as a single XML file was converted to a CSV file using a manually written script which contains all bug properties, previously mentioned aggregated properties but doesn't contain any of the comments.

This CSV file was later imported in the data mining software package that we used and each property was assigned a type as seen in the screenshot below. In this particular case shown in the screenshot, {\it Component} was selected as the label or class for classification.

%\includegraphics{s1-types} %todo convert from jpg to eps

Properties of type {\it text} will be later converted to vectors using the text processing features of the data mining software package.

% subsection Processing of the data (end)

% section Data (end)

\section{Tools} % (fold)
\label{sub:Tools}
We have used two open source data mining tools Weka and Rapid Miner.
Weka is tool which we both know from Computational linguistic course and its GUI can be used
for simple data or text mining tasks. Unfortunately, Weka GUI requires a lot of memory and contain a few bugs, 
which can be reduced by using command line interface.

On the other hand, Rapid Miner interface is very user friendly and it wraps a far more 
implementation of data mining algorithms than Weka. In fact, it contains Weka.

\subsection{Weka} % (fold)
\label{sub:Weka}

At first we used the tool Weka to familiarise ourselves with data mining algorithms.
It's main advantage compared to Rapid Miner is command line interface, where commands can be simply copied and pasted.
On contrary, the CLI is very confusing and iterative process of selecting correct arguments in data mining is really exhausting due to a large amount of processed data.

\subsection{Android bugs preprocessing for Weka} % (fold)
\label{sub:Android bugs preprocessing for Weka}
\begin{itemize}
    \item We created csv files
    \item We ran unsuccessfully command
\begin{lstlisting}[language=sh]
$ java weka.core.converters.CSVLoader ANDROID_PLATFORM_BUGS.csv > ANDROID_PLATFORM_BUGS.arff
\end{lstlisting}
    \item so we design our own script
    \item  We tried tried to convert strings attributes to vectors for 1000 most important words. Unfortunately, following command crashes.
\begin{lstlisting}[language=sh]
    java weka.filters.unsupervised.attribute.StringToWordVector -i android-platform-bugs.arff -o android_plat-bugs-Vectors.arff
\end{lstlisting}
    \item We found out that the script works if there only the string attributes so we generated
    android-plat-strings.arff and android-plat-except-strings.arff
    \item We split them manually on the data part and the header.
    The headers we simply concatenated. On the data we applied following commands.
\begin{lstlisting}[language=sh]
nl -s " " -b a android-plat-except-strings.data # number lines
nl -s " " -b a android-plat-strings.data # number lines
sed s:\(......\) :\1 a : -f *except* > number-and-prefix-a
sed s:\(......\) :\1 b : -f *strings* > number-and-prefix-b
cat *strings* *except*> mixed.data
sort -n data > sorted.data
sed s:\n...... b :: -f sorted.data> joined-related-lines
sed s:...... a :: -f joined-related-lines > without-prefix
cat mixed-head without-prefix > android-plat-bugs2vectors.arff
\end{lstlisting}
    \item Next day we realised, that weka crashed because our attributes {\it status}, {\it type} or {\it id} were also generated by {\bf StringToWordVector} commnand, so we have multiple attributes with the same names. So instead of splitting in previous steps we just rename our attributes.
    \item Finally, we could run our example. We have had experienced that Weka requires much less memory if is launch from command line, so we had to find out parameters for command line first.
        \begin{itemize}
            \item We launch Weka
            \item Switched to Explorer
            \item Set up {\it Component} as label attribute for training and testing
            \item Choose 10 fold cross validation
            \item Choose Support Vector Machines (SMO)
            \item Copy parameters generated by GUI to command line
        \end{itemize}
    \item We run command
\begin{lstlisting}[language=sh]
    java weka.classifiers.functions.SMO -C 1.0 -L 0.0010 -P 1.0E-12 -N 0 -V -1 -W 1 -K "weka.classifiers.functions.supportVector.PolyKernel -C 250007 -E 1.0" -t sample-vectorised.arff -c 7> smo.out
\end{lstlisting}
    \item After 3:34:12 we got precision of 79\% on sample containing 4000 items with following accuracy and confusion matrix.
    Actually, as you can see the results are quite good because most of the misclassified samples are from category null.
    It can be perceived like that our classifier tried to classify them, where the people forget about it.

\begin{tabular}{|c|c|c|c|c|c|c|c|c|c|c|c|c|c|c|c|c|}
a & b & c & d & e & f & g & h & i & j & k & l & m & n & o & p   &<-- classified as\\
45 & 9 & 1 & 1 & 1 & 0 & 1 & 0 & 0 & 0 & 0 & 0 & 0 & 0 & 0 & 0 &  a = Browser\\
5&2138 & 34 & 52 & 19 & 0 & 9 & 10 & 9 & 15 & 3 & 0 & 1 & 0 & 0 & 0 &  b = null\\
0 & 65 & 17 & 12 & 1 & 0 & 2 & 6 & 1 & 4 & 1 & 0 & 0 & 0 & 0 & 0 &  c = Framework\\
0 & 75 & 6 & 38 & 4 & 0 & 0 & 6 & 2 & 4 & 0 & 0 & 0 & 0 & 0 & 0 &  d = Applications\\
0 & 47 & 1 & 2 & 54 & 0 & 0 & 0 & 0 & 1 & 0 & 0 & 0 & 0 & 0 & 0 &  e = Tools\\
0 & 2 & 0 & 0 & 0 & 0 & 0 & 0 & 0 & 0 & 0 & 0 & 0 & 0 & 0 & 0 &  f = User\\
1 & 18 & 1 & 0 & 1 & 0 & 32 & 0 & 0 & 0 & 0 & 0 & 1 & 0 & 0 & 0 &  g = Dalvik\\
1 & 38 & 4 & 5 & 0 & 0 & 0 & 18 & 0 & 0 & 0 & 0 & 0 & 0 & 0 & 0 &  h = Google\\
0 & 35 & 5 & 3 & 0 & 0 & 0 & 0 & 2 & 0 & 0 & 0 & 0 & 0 & 0 & 0 &  i = GfxMedia\\
0 & 33 & 3 & 10 & 1 & 0 & 0 & 2 & 0 & 15 & 0 & 0 & 0 & 0 & 0 & 0 &  j = Device\\
0 & 15 & 2 & 0 & 0 & 0 & 1 & 1 & 1 & 3 & 2 & 0 & 0 & 0 & 0 & 0 &  k = System\\
0 & 2 & 0 & 2 & 0 & 0 & 0 & 0 & 0 & 0 & 0 & 0 & 0 & 0 & 0 & 0 &  l = Build\\
0 & 14 & 2 & 0 & 2 & 0 & 0 & 0 & 0 & 1 & 0 & 0 & 4 & 0 & 0 & 0 &  m = Web\\
0 & 0 & 0 & 0 & 0 & 0 & 0 & 0 & 0 & 0 & 0 & 0 & 0 & 0 & 0 & 0 &  n = Market\\
0 & 0 & 0 & 0 & 0 & 0 & 0 & 0 & 0 & 0 & 0 & 0 & 0 & 0 & 0 & 0 &  o = Android\\
0 & 0 & 0 & 0 & 0 & 0 & 0 & 0 & 0 & 0 & 0 & 0 & 0 & 0 & 0 & 0 &  p = Docs\\
\end{tabular}

    \item We reran the experiment without null values in Component 
    \item In Another experiment we tried to deduce which owner should care about the bug.
    At first we ran the experiment with null values and got following results of Stratified 10 folds cross-validation

\begin{tabular}{|c|c|c|}
Correctly Classified Instances      & 3685  &            77.907  \\
Incorrectly Classified Instances    & 1045  &            22.093  \\
\end{tabular}

\begin{tabular}{|c|c|}
Kappa statistic                &         0.5353\\
Mean absolute error            &         0.019 \\
Root mean squared error        &         0.0972\\
Relative absolute error        &       184.3532 \\
Root relative squared error    &       136.7836\\
Total Number of Instances      &      4730   \\
\end{tabular}




\end{itemize}

% subsection Android bugs preprocessing for Weka (end)

% subsection Weka (end)

\subsection{Rapid Miner} % (fold)
\label{sub:Rapid Miner}

Our tool of choice was Rapid Miner because ...
simple and nice interface
no bugs
presenting possibilities via diagrams, graphs, etc.

% subsection Rapid Miner (end)

% section Tools (end)

\section{Classification} % (fold)
\label{sub:Classification}

\subsection{Classification of bugs by component} % (fold)
\label{sub:Classification of bugs bugs by component}

When a bug report is submitted, someone has to look at it and determine which component it affects and assigned an appropriate label. Currently it is done manually by one of the maintainers. We wondered if it's possible to use data mining to automatically classify new bugs by component.

We came to the conclusion that only title and description should be used to determine the component since they describe the bug but other properties are just metadata. The outcome of the classifier should be the component name that this bug affects.

Only the bugs which already had a component label assigned to them were chosen as the training data for data mining because the bugs with a component assigned to them are already classified and we didn't have to do it manually.

Title and description values were transformed into lower case and then tokenized using non letters as delimiters and then word vectors were generated from them using TF-IDF. It's worth pointing out that if a more sophisticated tokenization algorithm was used, perhaps it would significantly improve the accuracy of classifiers.

After the data was pre-processed, we chose various classifying algorithms that were learned about in class and compared their accuracy. Accuracy is used to assess the performance of classifiers. It was mentioned in the lectures that a stratified 10-fold cross validation is recommended for estimating accuracy. Because we wanted to try out various algorithms and compare them, we only used a sample of data because it would take too much time to the classifiers to run on our desktop computers.

\begin{tabular}{|c|c|c|c|c|}
\hline
Type     &       Algorithm   & Sample &  Time &  Accuracy   \\
\hline
\hline
Decision Trees   Weka J48    &  500  &   2:19 &    44.80\%  \\
Decision Trees   Weka J48    & 1000  &  10:25 &    45.50\%  \\
Decision Trees   Weka LAD-Tree &  500  &        &           \\
\hline
\end{tabular}

% subsection Classification of bugs bugs by component (end)

\subsection{Classification of bugs by developer} % (fold)
\label{sub:Classification of bugs bugs by developer}
% subsection Classification of bugs bugs by developer (end)

% section Classification (end)

\section{Prediction} % (fold)
\label{sub:Prediction}

\begin{tabular}{|c|c|c|c|c|}
\hline
Type             Algorithm         Sample    Time    Accuracy
\hline
\hline
neural network 2 layers (100,100)  & 50     &  17:54 & 172.611 MSE \\
polynomial regression              & 50     &  5:16  & 159974516.535  MSE  -> does not have shape of polynom\\
polynomial regression              & 500    &  6:23:16 & 2620470.535  MSE  -> does not have shape of polynom\\
\hline
\end{tabular}


\subsection{Prediction of bug lifespan} % (fold)
\label{sub:Prediction of bug lifespan}
% subsection Prediction of bug lifespan (end)

\section{Clustering} % (fold)
\label{sub:Clustering}
% section Clustering (end)

\section{Test} % (fold)
\label{sec:Test}

% section Test (end)
