The main purpose of this milestone is to try out various data mining techniques.

We decided to explore the following data mining techniques in this milestone: classification, prediction and clustering. For each technique we evaluated various algorithms mentioned in the lectures and compared them to some others that were available to use in the data mining software package that we used. 

We also extracted some new information from the models built during the mining process to show some things that we found interesting which is also the goal of the MSR 2012 challange.

\section{Data} % (fold)
\label{sub:Data}
We chose to mine the Android bug database because it was the data was already provided by the organizers of the MSR 2012 challange. More specifically, we chose to focus on the bug reports in this database as we believe it is the most important part of the entire database and it should give the most interesting results. The database contains 20169 reported bugs (up to December 3, 2011). 

\subsection*{Description of the data} % (fold)
\label{sub:Description of the data}

Here is the available data about the bugs.

\begin{itemize}
\item Bug ID - unique identifier assigned to each bug
\item Title - short description of the bug
\item Description - detailed description of the bug without a predefined structure
\item Type - type of the bug: Defect (bug report), Enhancement (feature request)
\item Status - current status of the bug (Reviewed, New, Duplicate, Declined, NeedsInfo, FutureRelease, Released, Spam, Unreproducible, Question, WorkingAsIntended, Assigned, Unassigned, UserError)
\item Priority - urgency of the bug to be fixed (Small, Medium, High, Critical, Blocker)
\item Component - category of the bug based on which part of the system the bug affects
\item Stars - number of people that have starred this bug (could be an indicator of popularity)
\item Owner - assigned developer to fix  the bug
\item Reported By - person who reported the bug
\item Opened Date - when the bug was reported
\item Closed On - when the bug was marked as closed
\end{itemize}

It is also possible to aggregate the comments for a bug to determine additional data about it.

\begin{itemize}
\item Comment count - the number of comments for this bug
\item Number of commenters - the number of distinct people participating in the discussion about the bug
\item How long it took the solve the bug - by subtracting the close date from the open date
\end{itemize}

Title, description, type are entered by the user. System also automatically generated the bug id, marks it as a new bug, assigns a medium priority to the bug, makes the user as the reporter of the bug and sets the opened date to the date and time of the submition.

Component and owner are later manually assigned by one of the administrators. Status, priority can later be changed, however, these changes are not reflected in the available data, only the latest state of the bug is available.
% subsection Description of the data (end)

\subsection*{Processing of the data} % (fold)

The bug database which was provided as a single XML file was converted to a CSV file using a manually written script which contains all bug properties, previously mentioned aggregated properties but doesn't contain any of the comments.

This CSV file was later imported in the data mining software package that we used and each property was assigned a type as seen in the screenshot below. In this particular case shown in the screenshot, *Component* was selected as the label or class for classification.

*s1\_types.png*

Properties of type *text* will be later converted to vectors using the text processing features of the data mining software package.

% subsection Processing of the data (end)

% section Data (end)

\section{Tools} % (fold)
\label{sub:Tools}

\subsection{Weka} % (fold)
\label{sub:Weka}

At first we used the tool Weka to familizarlize ourselves with ....

% subsection Weka (end)

\subsection{Rapid Miner} % (fold)
\label{sub:Rapid Miner}

Our tool of choice was Rapid Miner because ...

% subsection Rapid Miner (end)

% section Tools (end)

\section{Classification} % (fold)
\label{sub:Classification}

\subsection{Classification of bugs by component} % (fold)
\label{sub:Classification of bugs bugs by component}

When a bug report is submitted, someone has to look at it and determine which component it affects and assigned an appropriate label. Currently it is done manually by one of the maintainers. We wondered if it's possible to use data mining to automatically classify new bugs by component.

We came to the conclusion that only title and description should be used to determine the component since they describe the bug but other properties are just metadata. The outcome of the classifier should be the component name that this bug affects.

Only the bugs which already had a component label assigned to them were chosen as the training data for data mining because the bugs with a component assigned to them are already classified and we didn't have to do it manually.

Title and description values were transformed into lower case and then tokenized using non letters as delimiters and then word vectors were generated from them using TF-IDF. It's worth pointing out that if a more sophisticated tokenization algorithm was used, perhaps it would significantly improve the accuracy of classifiers.

After the data was pre-processed, we chose various classifying algorithms that were learned about in class and compared their accuracy. Accuracy is used to assess the performance of classifiers. It was mentioned in the lectures that a stratified 10-fold cross validation is recommended for estimating accuracy. Because we wanted to try out various algorithms and compare them, we only used a sample of data because it would take too much time to the classifiers to run on our desktop computers.

Type             Algorithm     Sample    Time    Accuracy
Decision Trees   Weka J48       500      2:19      44.80%
Decision Trees   Weka J48      1000     10:25      45.50%
Decision Trees   Weka LAD-Tree  500       

% subsection Classification of bugs bugs by component (end)

\subsection{Classification of bugs by developer} % (fold)
\label{sub:Classification of bugs bugs by developer}
% subsection Classification of bugs bugs by developer (end)

% section Classification (end)

\section{Prediction} % (fold)
\label{sub:Prediction}

\subsection{Prediction of bug lifespan} % (fold)
\label{sub:Prediction of bug lifespan}
% subsection Prediction of bug lifespan (end)

\subsection{Prediction of bug rating} % (fold)
\label{sub:Prediction of bug rating}
% subsection Prediction of bug rating (end)

% section Prediction (end)

\section{Clustering} % (fold)
\label{sub:Clustering}
% section Clustering (end)

\section{Test} % (fold)
\label{sec:Test}

% section Test (end)
